\documentclass[12pt,]{article}
\usepackage{lmodern}
\usepackage{amssymb,amsmath}
\usepackage{ifxetex,ifluatex}
\usepackage{fixltx2e} % provides \textsubscript
\ifnum 0\ifxetex 1\fi\ifluatex 1\fi=0 % if pdftex
  \usepackage[T1]{fontenc}
  \usepackage[utf8]{inputenc}
\else % if luatex or xelatex
  \ifxetex
    \usepackage{mathspec}
  \else
    \usepackage{fontspec}
  \fi
  \defaultfontfeatures{Ligatures=TeX,Scale=MatchLowercase}
    \setmainfont[]{Times New Roman}
\fi
% use upquote if available, for straight quotes in verbatim environments
\IfFileExists{upquote.sty}{\usepackage{upquote}}{}
% use microtype if available
\IfFileExists{microtype.sty}{%
\usepackage{microtype}
\UseMicrotypeSet[protrusion]{basicmath} % disable protrusion for tt fonts
}{}
\usepackage[margin=2.54cm]{geometry}
\usepackage{hyperref}
\hypersetup{unicode=true,
            pdftitle={Insert title of project here},
            pdfauthor={Insert names of team members here},
            pdfborder={0 0 0},
            breaklinks=true}
\urlstyle{same}  % don't use monospace font for urls
\usepackage{graphicx,grffile}
\makeatletter
\def\maxwidth{\ifdim\Gin@nat@width>\linewidth\linewidth\else\Gin@nat@width\fi}
\def\maxheight{\ifdim\Gin@nat@height>\textheight\textheight\else\Gin@nat@height\fi}
\makeatother
% Scale images if necessary, so that they will not overflow the page
% margins by default, and it is still possible to overwrite the defaults
% using explicit options in \includegraphics[width, height, ...]{}
\setkeys{Gin}{width=\maxwidth,height=\maxheight,keepaspectratio}
\IfFileExists{parskip.sty}{%
\usepackage{parskip}
}{% else
\setlength{\parindent}{0pt}
\setlength{\parskip}{6pt plus 2pt minus 1pt}
}
\setlength{\emergencystretch}{3em}  % prevent overfull lines
\providecommand{\tightlist}{%
  \setlength{\itemsep}{0pt}\setlength{\parskip}{0pt}}
\setcounter{secnumdepth}{5}
% Redefines (sub)paragraphs to behave more like sections
\ifx\paragraph\undefined\else
\let\oldparagraph\paragraph
\renewcommand{\paragraph}[1]{\oldparagraph{#1}\mbox{}}
\fi
\ifx\subparagraph\undefined\else
\let\oldsubparagraph\subparagraph
\renewcommand{\subparagraph}[1]{\oldsubparagraph{#1}\mbox{}}
\fi

%%% Use protect on footnotes to avoid problems with footnotes in titles
\let\rmarkdownfootnote\footnote%
\def\footnote{\protect\rmarkdownfootnote}

%%% Change title format to be more compact
\usepackage{titling}

% Create subtitle command for use in maketitle
\providecommand{\subtitle}[1]{
  \posttitle{
    \begin{center}\large#1\end{center}
    }
}

\setlength{\droptitle}{-2em}

  \title{Insert title of project here}
    \pretitle{\vspace{\droptitle}\centering\huge}
  \posttitle{\par}
  \subtitle{Web address for GitHub repository}
  \author{Insert names of team members here}
    \preauthor{\centering\large\emph}
  \postauthor{\par}
    \date{}
    \predate{}\postdate{}
  

\begin{document}
\maketitle

\newpage

\textless{}Arrow brackets are used for annotating the RMarkdown files.
Text within these brackets should not appear in the final version of the
PDF document\textgreater{}

\textless{}\textbf{General Guidelines}\textgreater{} \textless{}1. Write
in scientific style\textgreater{} \textless{}2.
\href{https://rmarkdown.rstudio.com/lesson-3.html}{Global options for R
chunks} should be set so that only relevant output is
displayed\textgreater{} \textless{}3. Make sure your final knitted PDF
looks professional. Format tables appropriately, size figures
appropriately, make sure bulleted and numbered lists appear as such,
avoid awkwardly placed page breaks, etc.\textgreater{}

\hypertarget{rationale-and-research-questions}{%
\section{Rationale and Research
Questions}\label{rationale-and-research-questions}}

\textless{}Write 1-2 paragraph(s) detailing the rationale for your
study. This should include both the context of the topic as well as a
rationale for your choice of dataset (reason for location, variables,
etc.) A few citations should be included to give context for your topic.
You may choose to configure autoreferencing for your citations or add
these manually.\textgreater{}

\textless{}At the end of your rationale, introduce a numbered list of
your questions (or an overarching question and sub-questions). Each
question should be accompanied by one or more working hypotheses,
inserted beneath each question.\textgreater{}

\newpage

\hypertarget{dataset-information}{%
\section{Dataset Information}\label{dataset-information}}

\textless{}Provide information on how the dataset for this analysis were
collected, the data contained in the dataset, and any important pieces
of information that are relevant to your analyses. This section should
contain much of same information as the metadata file for the dataset
but formatted in a way that is more narrative.\textgreater{}

\textless{}Describe how your team wrangled your dataset in a format
similar to a methods section of a journal article.\textgreater{}

\textless{}Add a table that summarizes your data structure (variables,
units, ranges and/or central tendencies, data source if multiple are
used, etc.). This table can be made in markdown text or inserted as a
\texttt{kable} function in an R chunk. If the latter, do not include the
code used to generate your table.\textgreater{}

\newpage

\hypertarget{exploratory-analysis}{%
\section{Exploratory Analysis}\label{exploratory-analysis}}

\textless{}Insert exploratory visualizations of your dataset. This may
include, but is not limited to, graphs illustrating the distributions of
variables of interest and/or maps of the spatial context of your
dataset. Format your R chunks so that graphs are displayed but code is
not displayed. Accompany these graphs with text sections that describe
the visualizations and provide context for further
analyses.\textgreater{}

\textless{}Each figure should be accompanied by a caption, and each
figure should be referenced within the text\textgreater{}

\newpage

\hypertarget{analysis}{%
\section{Analysis}\label{analysis}}

\textless{}Insert visualizations and text describing your main analyses.
Format your R chunks so that graphs are displayed but code and other
output is not displayed. Instead, describe the results of any
statistical tests in the main text (e.g., ``Variable x was significantly
different among y groups (ANOVA; df = 300, F = 5.55, p \textless{}
0.0001)''). Each paragraph, accompanied by one or more visualizations,
should describe the major findings and how they relate to the question
and hypotheses. Divide this section into subsections, one for each
research question.\textgreater{}

\textless{}Each figure should be accompanied by a caption, and each
figure should be referenced within the text\textgreater{}

\hypertarget{question-1}{%
\subsection{Question 1: }\label{question-1}}

\hypertarget{question-2}{%
\subsection{Question 2:}\label{question-2}}

\newpage

\hypertarget{summary-and-conclusions}{%
\section{Summary and Conclusions}\label{summary-and-conclusions}}

\textless{}Summarize your major findings from your analyses in a few
paragraphs. What conclusions do you draw from your findings? Relate your
findings back to the original research questions and
rationale.\textgreater{}

\newpage

\hypertarget{references}{%
\section{References}\label{references}}


\end{document}
